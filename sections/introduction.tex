\section{Introduction}
The path planning problem for mobile robots can be defined as the search for a path which a robot (with specified geometry) has to follow in a described environment, in order to reach a particular position and orientation B, given an initial position and orientation. ~\cite{Buniyamin_2011_IJSAED}.

Path planning can be divided into two categories - global and local planning. If the knowledge of the environment is known beforehand, the global path can be planned before the robot starts to move. If there exists no prior knowledge about the environment, the path must be computed online and the agent must avoid obstacles in real-time ~\cite{Buniyamin_2010_ICOSSSE}.

In the past year, neural network based models have shown remarkable results in a sensory role for obtaining depth information from standard RGB camera images ~\cite{Liu_2015_CVPR}.

This paper describes an approach for an agent to navigate from source co-ordinates to destination co-ordinates while avoiding obstacles that may appear on it's path. Through this project, we also introduce an autonomous control policy to govern aerial agents.

