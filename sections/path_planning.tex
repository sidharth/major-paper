\subsection{Path Planning}
The entire 3 dimensional environment is divided into N unit volume cubes, centred at integer coordinates.
 
A modified A* algorithm is used as the basis for path planning. It is augmented by the depth map discussed in the previous section. The interfacing between the algorithm and the generated depth map will be discussed in the subsequent section.

The following implicit data structures and objects are maintained by the agent controller object:
 
\textbf{set\_visited}
A set of previously visited nodes
 
\textbf{set\_prospects}
The currently available prospects for immediate motion
 
\textbf{stack\_active}
A stack of nodes on the currently active path
 
\textbf{to\_cube}
The highest priority cube in the immediate surroundings on the path to destination

Two types of step movement policies were examined for this control policy:
\begin{enumerate}
\item 4-connected policy
\item 8-connected policy
\end{enumerate}

Two distance heuristics were compared in the experiment:
\begin{enumerate}
\item Manhattan distance
\item Euclidean distance
\end{enumerate}

\subsection{Interfacing depth map with path planner}


\begin{algorithm}
\caption{Calculate $y = x^n$}
\label{alg1}
\begin{algorithmic}[1]
\STATE Import Blender model, and obstacles from the Shapenet Database
\STATE Two cubes are selected
\STATE Ensure there is no collision with src\_cube or dest\_cube
\STATE Initialize agent controller with these initial values
\STATE Interface agent controller with left and right camera stream
\STATE Start game and log the step counts
\STATE End log when destination reached
\end{algorithmic}
\end{algorithm}

\begin{figure}
  \includegraphics[width=\linewidth]{images/flowchart.png}
  \caption{Interfacing depth map with path planner}
  \label{fig:flowchart1}
\end{figure}