\section{Related Work}
Synthetic generated imagery is a useful tool in computer vision as it allows for generation of a large number of arbritrary examples in various positions and orientations. This would allow a more precise sample set than would ordinarily be possible with real world data. It also provides us with precise groundtruth data such as depth information which might not be trivial to collect from real data.

Synthetically generated data has been used in several instances in the past to augment performance. Taylor et al. present a system called Object Video Virtual Video (OVVV) ~\cite{Taylor_2007} based on Half-life for evaluation of tracking in surveillance systems. 
 
In the recent literature, Peng et al. , and Sun and Saenko use synthetic CAD generated images models to fine-tune on the object detection task and significantly outperform previous methods with this approach. 

Lim et al. , and Aubry et al. use CAD models for detection and object alignment in the image. Aubry and Russell use synthetic RGB images rendered from CAD models to analyze the response pattern and the behavior of neurons in the commonly used deep convolutional networks.

There has recently been increased interest in using video game data to train computer vision models ~\cite{Shafaei_LS16}. Although video games generate images from a finite set of textures, there is variation in viewpoint, illumination, weather, and level of detail which could provide valuable augmentation of the data. 

using synthetic data
depth estimation
neural networks for depth estimation
path planning
localised path planning