\begin{abstract} 
This paper presents an approach for navigating an autonomous agent in a simulated room environment with multiple obstacles. It presents a technique for depth map estimation using stereoscopic images - a vital intermediate technique for autonomous navigation and pathfinding. It also proposes a path planner interfaced with the depth map for locally planning a sequence of steps to the desired location. A single RGB camera does not contain sufficient information to estimate depth with satisfactory accuracy. Depth cameras are significantly more expensive than RGB cameras, and hence would not be suited for projects requiring lower budget constraints. Hence a middle ground is required which requires depth calculation from limited available input. Depth estimation is accomplished with a convolutional neural network trained on synthetic randomly generated stereo image data. The path planner comprises a short-term path planner for obstacle avoidance, and a long-term path planner for approaching the destination co-ordinates. The results show that this approach yields a factor of 1.8 times the steps of human performance on the same task.
\end{abstract}
